\documentclass[11pt]{article}

\usepackage{amsmath}
\usepackage{textcomp}

% Add other packages here %


% Put your group number and names in the author field %
\title{\bf Excercise 3\\ Implementing a deliberative Agent}
\author{Group \textnumero : Student 1, Student 2}


% N.B.: The report should not be longer than 3 pages %


\begin{document}
\maketitle

\section{Model Description}

\subsection{Intermediate States}
% Describe the state representation %
One state is represented by: 
\begin{itemize}
\item \verb|currentCity| the current city of the vehicle
\item \verb|remainingTasks| the remaining tasks to pickup
\item \verb|currentTasks| the tasks currently carried by the vehicle 
\end{itemize}

\subsection{Goal State}
% Describe the goal state %
The goal state are the states for which  \verb|remainingTasks| and \verb|currentTasks| are empty

\subsection{Actions}
% Describe the possible actions/transitions in your model %
Actions are either going to pickup a remaining task if its weight is less than the available space in the trunk, or delivering a current task.

\section{Implementation}

\subsection{BFS}
% Details of the BFS implementation %
We implement BFS using a linked list. We decided to keep track of nodes of the tree which have the end up in the same state (\verb|currentCity|, \verb|remainingTasks|and \verb|currentTasks|) but with different costs and keep only the one with the lowest cost.

\subsection{A*}
% Details of the A* implementation %
We implement \verb|A*| using a priority queue, first with the heuristic function $h(x) = 0$, then with two other heuristic functions detailed bellow.

\subsection{Heuristic Function}
% Details of the heuristic functions: main idea, optimality, admissibility %
\begin{itemize}
\item $h$ is the minimum of the distance of several randomly picked path, multiplied by a discount. This one is not guaranteed to be admissible and optimal.
\item $h$ is the maximum over all remaining packages of the distance to pickup and deliver this package. this heuristic is admissible and optimal.
\end{itemize}

\section{Results}

\subsection{Experiment 1: BFS and A* Comparison}
% Compare the two algorithms in terms of: optimality, efficiency, limitations %
% Report the number of tasks for which you can build a plan in less than one minute %

\subsubsection{Setting}
% Describe the settings of your experiment: topology, task configuration, etc. %
We use the topology \verb|config/topology/switzerland.xml|. And the second heuristic for ASTAR.

\subsubsection{Observations}
% Describe the experimental results and the conclusions you inferred from these results %
Our implementation of ASTAR can compute an optimal solution in less than one minute for 7 tasks, but BFS can do it only for 6 tasks. They both give optimal results.

\subsection{Experiment 2: Multi-agent Experiments}
% Observations in multi-agent experiments %

\subsubsection{Setting}
% Describe the settings of your experiment: topology, task configuration, etc. %
We ran two ASTAR agents on the topology \verb|config/topology/switzerland.xml|.

\subsubsection{Observations}
% Describe the experimental results and the conclusions you inferred from these results %
We observe that the agents have several conflicts and even though they achieve the whole delivery in less time, they do not have an optimal collaboration.

\end{document}