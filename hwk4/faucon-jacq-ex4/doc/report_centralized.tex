\documentclass[11pt]{article}

\usepackage{amsmath}
\usepackage{textcomp}
\usepackage{enumitem}

% Add other packages here %


% Put your group number and names in the author field %
\title{\bf Excercise 4\\ Implementing a centralized agent}
\author{Group \textnumero 10: Alexis Jacq, Louis Faucon}


% N.B.: The report should not be longer than 3 pages %


\begin{document}
\maketitle

\section{Solution Representation}

\subsection{Variables}
% Describe the variables used in your solution representation %
We define a solution as list of sequences of tasks to pickup/delivery, one sequence per vehicle, with $N_T$ tasks in total. In order to encode the different tasks, we associate the $i$-th pickup task $t_p^i$ with the integer $i$ (from $0$ to $N_T-1$), and the corresponding delivery task $t_d^i$ with the integer $i+N_T$ (from $N_T$ to $2N_T-1$). \\
\\
For example, the sequence $(1;\,\,5;\,\,1+N_T;\,\,5+N_T)$ means that the vehicle must pickup task 1, then pickup task 5, then deliver task 1, then deliver task 5.\\

\subsection{Constraints}
% Describe the constraints in your solution representation %
A solution respects constraints if both of the following conditions are verified:\\
\begin{itemize}[noitemsep,nolistsep]
\item all vehicles can realize its sequence of tasks, \textit{i.e.} all biggest sums of simultaneous carried tasks in sequences are smaller than corresponding vehicle's capacity.\\
\item all tasks are picked up at any time $i$ by one single vehicle and delivered at time $j>i$ by the same vehicle.
\end{itemize}

\subsection{Objective function}
% Describe the function that you optimize %
We want to optimize the total cost $C_\textbf{tot}(S)$ of a solution $S$:
$$C_\textbf{tot}(S) = \sum\limits_{v \in V} ( \textit{cost\_per\_km}(v) \sum\limits_{t \in tasks(v)} \textit{dist}\left[prev(v,t),\textit{city(t)}\right] )$$
where:
\begin{itemize}[noitemsep,nolistsep]
\item $prev(v,t))$ is the previous city in which the vehicle is before the task $t$ in the sequence $s$
\item \textit{cost\_per\_km}(s) is the cost per km of the vehicle to which is associated the sequence $s$.
\end{itemize}

\section{Stochastic optimization}


\subsection{Initial solution}
% Describe how you generate the initial solution %

At the beginning, the vehicle with the largest capacity is assigned to pickup and deliver all the tasks one by one:\\
\\
- sequence = $(0;\,\,0+N_T;\,\,...\,\,;\,\,N_T-1;\,\,2N_T-1)$ for the largest vehicle\\
- sequence = $\emptyset$ for all other vehicles\\
\\
This solution respects the constraints, otherwise one of the tasks has a weight larger than the largest capacity of available vehicles, and the problem has no solution.

\subsection{Generating neighbours}
% Describe how you generate neighbors %
At each step we generate a set of neighboring new solutions. This set originally contains the previous solution $S_\textit{old}$. We randomly choose a task $(t_p,t_d)$ to pickup and deliver which we remove from its vehicle $v_1$'s sequence, and a second vehicle $v_2$. We add to the set of neighboring solutions all the possible insertions of $t_p$ and $t_d$ into $v_2$'s sequence (such that $v_2$ can still realize its sequence given its capacity and such that $t_p$ appears before $t_d$). 

\subsection{Stochastic optimization algorithm}
% Describe your stochastic optimization algorithm %
With probability $p = 1/\sqrt{\tau}$ -- where $\tau$ is a temperature parameter linearly increasing with time --, we randomly select a neighbour in the generated set, and with probability $1-p$, we select the best neighboring solution (with the smaller cost). After 10000 iterations, we stop the algorithm.

\section{Results}

\subsection{Experiment 1: Model parameters}
% if your model has parameters, perform an experiment and analyze the results for different parameter values %

% Describe the settings of your experiment: topology, task configuration, number of tasks, number of vehicles, etc. %
% and the parameters you are analyzing %
Our only parameters are the initial temperature $\tau_0$ and its increasing rate $\delta_\tau$. It controls how fast we trust a locally best solution rather than exploring random modifications. We cross-tested different values for each parameters: $\tau_0\in \left\lbrace 1,3,5\right\rbrace$, $\delta_\tau\in \left\lbrace 0,0.1,0.5,1,5 \right\rbrace$.\\
\\
We set the random seed at 12345, the number of tasks $N_T$ at 30 with constant weight $w=3$, the number of vehicles $N_V$ at 4 with capacities {3,6,9,12}.

% Describe the experimental results and the conclusions you inferred from these results %
We observe that if $\tau_0$ and $\delta_\tau$ are low, the solution always changes but keep big costs, and if they are too high, the algorithm stops too quickly into local optima, which could have been better with further explorations. We obtained the best results with $\tau_0=1$ and $\delta_\tau=0.5$. The best solution found uses several vehicles

\subsection{Experiment 2: Different configurations}
% Run simulations for different configurations of the environment (i.e. different tasks and number of vehicles) %

% Describe the settings of your experiment: topology, task configuration, number of tasks, number of vehicles, etc. %
Random seed 33333, $N_T=50$ with constant weight $w=3$, $N_V=8$ with capacities {3,3,6,6,9,9,12,12}. In this larger setup, we observed the best results with an even smaller increasing rate $\delta_\tau=0.1$. 

\subsection{Experiment 3: Edge case}
% Run simulations for different configurations of the environment (i.e. different tasks and number of vehicles) %

% Describe the settings of your experiment: topology, task configuration, number of tasks, number of vehicles, etc. %
Random seed 12345, $N_T=30$ with constant weight $w=3$, $N_V=4$ with cost per km of {100,100,100,1}. We observe that our algorithm finds converges to give all the tasks to the vehicle with lower cost per km, but this convergence takes many iterations.

\subsection{Complexity}
At each iteration our algorithm creates a number of neighbours that can be in the order of$O(N_T^2)$. The worst case complexity at each iteration does not depend on the number of vehicle or on other parameters of the system, however these parameter will influence the quality of the final solution.


\end{document}
